//////////////// Constants ////////////////
var RGB_VIS = {min: 0, max: 0.3, gamma: 1.5, bands: ['B4', 'B3', 'B2']};



//////////////// Functions ////////////////
function addNDVI(img) {
  var ndvi = img.normalizedDifference(['B5', 'B4']); // Landsat 8
  return img.addBands(ndvi.rename('ndvi')); //adds our new band directly back to the image
}

   
//////////////// Analysis ////////////////
var filtered = L8.filterDate('2016-05-01', '2016-08-30'); // this is called an image collection


// now map() our function across the image collection
var with_ndvi = filtered.map(addNDVI); //note this is the mathematical definition of 'map' ie apply this function across a set of things. it's not actuall cartography!


//////////////// User Interface ////////////////
Map.addLayer(with_ndvi.median(), RGB_VIS, 'RGB'); //.median() specifies the median image. the default is the plot the most recent in the collection
Map.addLayer(with_ndvi.median(), {bands:'ndvi', min:0, max:1, palette:'black,lightgreen'}, 'NDVI');
// try adding palette:'black,lightgreen' to change the viz params
//also, try looking around in the Inspector tab. Inspector is your friend for working with temporal data!
